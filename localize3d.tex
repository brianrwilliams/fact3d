\documentclass[11pt]{amsart}

\usepackage{macros,amsaddr,physics}

\addbibresource{fact3d.bib}

\renewcommand{\op}{\operatorname}
\newcommand{\RS}{\cR \cS}
\newcommand{\penrose}{\slashed{P}}

%\linespread{1.2} %for editing
%\usepackage{mathpazo}

\begin{document}

\title{Localization for factorization algebras in three dimensions}
\maketitle

\section{Introduction}




Let $\op{Fact}_{\cN=4}^{suco}$ be the category of $\cN=4$ superconformal factorization algebras on~$\R^3$.

\begin{thm}
This construction defines a (adjectives) functor
\beqn
\op{Fact}_{\cN=4}^{suco} \to \op{Alg} 
\eeqn
from the category of $\cN=4$ superconformal factorization algebras to the category of associative (really $A_\infty$) algebras.
\end{thm}

\brian{Maybe we have our main result for just the $R$-symmetry version of $\cN=4$. Let's see....}

\section{Supersymmetric structures on factorization algebras}

\subsection{Supersymmetric stress tensors as local Lie algebras}

I want to upgrade \cite[section 4]{DumiSeiberg} to local Lie algebras.
They only do $\cN=2$, but we should be able to just reduce their description of $\cN=4$ in four dimensions.

Let $\lie{S}_{\cN=2}$ be the local super Lie algebra underlying the $\cN=2$ stress tensor.
Let $\lie{R}_{\cN=2}$ be the local super Lie algebra underlying the $\cN=2$ $R$-symmetry multiplet.
Let $\lie{J}_{\cN=2}$ be the local super Lie algebra underlying the $\cN=2$ superconformal multiplet.
These local Lie algebras are defined on the site of spin Riemannian three-manifolds.

Let $\lie{J}_{\cN=2}$ be the following complex of super vector bundles on the site of three-dimensional Riemannian spin manifolds:
\begin{equation}
    \begin{tikzcd}
        & 0 & 1 \\ \hline
        even &C^\infty \ar[r,"\d"] &  \Omega^1_j \\
        even & \Vect \ar[r,"L_{(-)} g"] & \Gamma\left(\Sym^2(\T^*)_0\right)_{T} \\
        odd & \Gamma(\cS \otimes \C^2) \ar[r,"\penrose"] & \Gamma(\RS \otimes \C^2)_{\psi}
    \end{tikzcd}
\end{equation}

\begin{lem}
\label{lem:lieN=2}
There are maps of local super Lie algebras
\beqn
\lie{S}_{\cN=2} \to \lie{R}_{\cN=2} \to \lie{J}_{\cN=2} .
\eeqn
\end{lem}

Let $\lie{R}_{\cN=2}$ be the following complex of super vector bundles on the site of three-dimensional Riemannian spin manifolds:
\begin{equation}
    \begin{tikzcd}
        & 0 & 1 \\ \hline
        even &C^\infty \ar[r,"\d"] &  \Omega^1_j \\
        even & C^\infty \ar[r,"\d"] & \Omega^1_{\d^{-1}F} \\
        even & & C^\infty_{\d^{-1} H} \\
        even & \Vect \ar[r,"L_{(-)} g"] & \Gamma\left(\Sym^2(\T^*)\right)_T \\
        odd & \Gamma(\cS) \ar[r,"\nabla"] & \Omega^1 (\cS)
    \end{tikzcd}
\end{equation}

\subsection{Supersymmetry for factorization algebras}

\begin{dfn}
Let $\cF$ be a factorization algebra defined on the site of three-dimensional Riemannian spin manifolds.
\begin{itemize}
\item A $\cN=2$ \defterm{supersymmetric structure} on $\cF$ is a map of factorization algebras
\beqn
\UU \lie{S}_{\cN=2} \to \cF .
\eeqn
\item A $\cN=2$ \defterm{supersymmetric structure with $R$-symmetry} on $\cF$ is a map of factorization algebras
\beqn
\UU \lie{R}_{\cN=2} \to \cF .
\eeqn
\item A $\cN=2$ \defterm{superconformal structure} on $\cF$ is a map of factorization algebras
\beqn
\UU \lie{J}_{\cN=2} \to \cF .
\eeqn
\end{itemize}
\end{dfn}

From lemma \ref{lem:lieN=2}, and functoriality of the factorization enveloping algebra, we see that any $\cN=2$ superconformal structure induces a $\cN=2$ supersymmetric structure with $R$-symmetry.
Likewise, by forgetting the $R$-symmetry any $\cN=2$ supersymmetric structure with $R$-symmetry induces a $\cN=2$ supersymmetric structure.

\section{The $HT$ twist}

Let $Q_{HT}$ be a holomorphic-topological supercharge associated to a THF structure.

\begin{thm}
There is a quasi-isomorphism of local Lie algebras on the site of THF manifolds
\beqn
\left(\lie{J}_{\cN=2}\right)^{Q_{HT}} \simeq \cA^\bu(\T) 
\eeqn
\end{thm}

Let $\T^{1|1}$ be the following complex of THF vector bundles
\beqn
\begin{tikzcd}
\ul{-1} & \ul{0} & \ul{1} \\
\cO_{f'} & \T_X & \T_{X'} \\
& \cO_{f} & .
\end{tikzcd}
\eeqn
We will denote sections in the top line by $(f, X, X')$ and sections in the bottom line by $f'$.
This complex of sheaves has the structure of a Lie algebra with brackets defined by
\begin{itemize}
\item The vector fields $X$ in degree zero act on all other sections via the Lie derivative.
\item For $f' \in \cO$ of degree zero we have
\beqn
[f', f] = - f' f , \quad [f',X'] = f' X' .
\eeqn
\item Finally, there is the bracket
\beqn
[X',f] = (X' \cdot f)_{f'} + (f X')_{X} .
\eeqn
\end{itemize}
These brackets endow $\cA^\bu(\T^{1|1})$ with the structure of a local Lie algebra.

\begin{thm}
There is a quasi-isomorphism of local Lie algebras on the site of THF manifolds
\beqn
\left(\lie{J}_{\cN=4}\right)^{Q_{HT}} \simeq \cA^\bu(\T^{1|1}) . 
\eeqn
\end{thm}

\section{Holomorphic-topological factorization algebras}

\begin{dfn}
Let $M$ be a THF three-manifold and let $\cF$ be a factorization algebra on $M$.
A \defterm{THF structure} on $\cF$ is a map of factorization algebras
\beqn
\UU \cT \to \cF .
\eeqn 
\end{dfn}

\begin{prop}
\end{prop}

\section{Free hyper}

Suppose that $V$ is a vector space with a linear symplectic structure $\omega \in \wedge^2 V^*$.
From $V$ we can associate an $\EE_1$-chiral algebra $\cA_V$ whose underlying graded vector space is
\beqn
\cA_V = \cO\left(V[[z]][\ep]\right) .
\eeqn
Here, $\ep$ is a formal variable of cohomological degree one.
In physics terminology, $\cA_V$ is the algebra of local operators in the minimal twist of the $\cN=4$ hypermultiplet with values in $V$.
We can forget the chiral structure and view $\cA_V$ simply as an associative algebra.
This associative algebra structure is (graded) commutative, being the obvious one on the (graded) symmetric algebra of polynomials on $V[[z]][\ep]$.

Notice that $V[[z]][\ep] = \cO(D^{1|1}) \otimes V$ is a representation of the Lie algebra of super/graded vector fields on the super/graded disk $D^{1|1}$ in the way that vector fields act as functions.
Let $\cA^S_V$ be the deformation of $\cA_V$ with respect to the Maurer--Cartan element $z \frac{\del}{\del \eps}$.

On to the actual construction. 
Let $\cA$ be the de Rham--Dolbeault complex of $\R \times \C$.
Consider 
\beqn
\cL_V = \cA \otimes V^*[\ep][-1]
\eeqn
as an abelian local Lie algebra where $\ep$ is a formal variable of cohomological degree $1$.
The local Lie algebra $\cL_V$ is equipped with the following local cocycle
\beqn
\phi_\omega = \int_{\R \times \C} \omega^{-1}(\alpha, \alpha') ,
\eeqn
where $\omega^{-1}$ is the inverse symplectic form on $V^*$ and where we have written sections as $\alpha + \eps \alpha'$.
Let $\cA_V$ be the factorization algebra on $\R \times \C$ obtained as the enveloping factorization algebra of this local Lie algebra twisted by $\phi_\omega$
\beqn
\cA_V \define \U^{fact}_{\phi_\omega} \left(\cL_V \right) .
\eeqn
Explicitly, if $U \subset \R \times \C$ then this factorization algebra assigns the cochain complex
\beqn
\cA_V(I \times U) = \left(\Sym\left(\cA_c (U) \otimes V^* [\ep]\right) \; , \; \d_{dR} + \dbar + \phi_{\omega} \right) ,
\eeqn
where $\d_{dR}$ is the de Rham differential acting on forms on $\R$.

Let $\cL_V^S$ be the deformed local Lie algebra which we obtain from $\cL_V$ by deforming the differential
\beqn
\d_{dR} + \dbar \rightsquigarrow \d_{dR} + \dbar + S = \d_{dR} + \dbar + z \frac{\del}{\del \ep} .
\eeqn

\begin{lem}
$\phi_\omega$ is a local cocycle for $\cL_V^S$.
\end{lem}

Define the resulting deformed factorization algebra
\beqn
\cA^S_V \define \U^{fact}_{\phi_\omega} \left(\cL_V \right) .
\eeqn
Explicitly, if $U \subset \R \times \C$ then this factorization algebra assigns the cochain complex
\beqn
\cA^S_V(I \times U) = \left(\Sym\left(\cA_c (U) \otimes V^* [\ep]\right) \; , \; \d_{dR} + \dbar + S + \phi_{\omega} \right) .
\eeqn

Finally, we define the relevant one-dimensional factorization algebra.
Consider the abelian local Lie algebra $\Omega^\bu_\R \otimes V[-1]$ on $\R$ and the local cocycle
\beqn
\phi^{1d}_{\omega} = \int_{\R} \omega^{-1}(\alpha, \alpha) .
\eeqn
Let $\cA^{1d}_V$ be the twisted enveloping factorization algebra
\beqn
\cA^\star_V \define \U^{fact}_{\phi^{1d}_{\omega}} \left(\Omega^\bu_\R \otimes V [-1]\right) .
\eeqn
As an associative algebra one has $\cA_V^{1d} \simeq \cO_{\star}(V)$ equipped with its Moyal $\star$-product.

\begin{prop}
Let $\pi \colon \R \times \C \to \R$ be projection.
There is a quasi-isomorphism
\beqn
\cA_V^{\star} \xto{\simeq} \pi_* \cA_V^S 
\eeqn
of factorization algebras on $\R$.
\end{prop}
\begin{proof}
We first define the linearized map of cosheaves of
\beqn
\Phi \colon \Omega^\bu_{\R,c} \otimes V \to \pi_* \cA_{c} \otimes V ,
\eeqn
where the right hand side is equipped with the differential $\d_{dR} + \dbar + S$.
Suppose $I \subset \R$ is an interval and let $\alpha \in \Omega^\bu_c \otimes V$ be a $V$-valued compactly supported form on $I$.
Let $\rho$ be a radially symmetric bump function on $\C$ centered at $0$.
Define
\beqn
\Phi (\alpha) = \rho \pi^* \alpha - \ep \frac{\dbar \rho}{z} \pi^* \alpha \in \cA^\bu(\pi^{-1}(I)) \otimes V [\ep] .
\eeqn
It is clear that $\Phi$ commutes with the de Rham differential $\d_{dR}$.
On the other hand
\beqn
(\dbar + S) \Phi(\alpha) = (\dbar \rho) \pi^* \alpha - (\dbar \rho) \pi^* \alpha = 0 .
\eeqn
Thus, $\Phi$ is a cochain map.
We observe that $\Phi$ is a quasi-isomorphism of cosheaves (this is Serre dual to the usual Koszul resolution for $0 \in \C$).

We extend $\Phi$ to $\cA_V^{1d}(I)$ so that it is a map of underlying graded commutative algebras.
To see that $\Phi$ is a map of factorization algebras, we only need to verify that it intertwines the local cocycles $\phi_{\omega}$ and $\phi_{\omega}^{1d}$.
This is direct computation:
\begin{multline}
\int_{\R \times \C} \omega^{-1} \left(\rho \pi^* \alpha - \ep \frac{\dbar \rho}{z} \pi^* \alpha, \rho \pi^* \alpha - \ep \frac{\dbar \rho}{z} \pi^* \alpha\right) = \# \int_\R \omega^{-1}(\alpha, \alpha) .
\end{multline}
There's a number here that only changes the definition of $\phi^{1d}_\omega$, which is not that important here.
\end{proof}


\printbibliography

\end{document}